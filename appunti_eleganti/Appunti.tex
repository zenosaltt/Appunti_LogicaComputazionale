\documentclass [12pt, letterpaper]{article}
\begin{document}
	\title{Appunti di Logica computazionale}
	\author{Rovesti Alberto}
	\date{2024}
	
	\maketitle
	
	Lo scopo di questa materia è quello di formalizzare il processo logico definendo verità oggettive in modo da creare ragionamenti/ conclusioni non ambigui
	
	
	
	\section{Concetti base}
	\begin{itemize}
		\item \textbf{Percezione} :
		
		Attraverso i nostri sensi noi non percepiamo la totalità del mondo e intlre possiamo percepirlo distorto Es: illusioni ottiche, giochi di luce 
		
		\item \textbf{Concettualizzazione} :
		
		Una volta osservato il mondo ci costruiamo in mente dei concetti che descrivono questo mondo: es:
		ho visto 100 sedie , se vedio un altra sedia anche diversa da tutte quelle viste prima la riconosco perchè ho una rappresentazione concettuale di cosa una sedia sia. Questo porta ad errori perchè come noi concepiamo le cose può essere influenzato dal nostro linguaggio (es: esperimento schianto macchine - a delle persone è stato fatto vedere un incidente e dopo gli hanno chiesto cosa fosse successo all'auto, le persone che conducevano gli esperimenti hanno usato diverse parole con diverse paersone per porre la domanda (le macchine si sono colpite, si sono scontrate, si sono toccate, si sono schiantate) in base alla gravità del termine utilizzato la risposta era conseguentemente catastrofica )
		
		\item \textbf{Rappresentazione} :
		
		Quando vogliamo descrivere qualcosa a qualcuno commettiamo sempre degli errori di rappresentazione. 
		Esempio: ci ricordiamo male, abbiamo dei baias, memorie parziale e più in generale tralasciamo un infinità di dettagli (che non possiamo comunicare per via del fatto che la nostra memoria è limitata e per un vincolo linguistico)
		
		\item \textbf{Ragionamento} :
		
		Partendo che i nostri pensieri sono basati su questi 3 concetti precedenti la nostra mente ragiona ed elabora e può portare a conclusioni sbagliate 
		
	\end{itemize}
	
	\section{elementi fondanti}
	
\end{document}