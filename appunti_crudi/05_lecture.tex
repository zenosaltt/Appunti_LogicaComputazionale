\chptr{Lezione}
\marginpar{\minitoc}


\section{Funzione}
Oggi finiamo mapping tra descrizione linguistica
e modello formale.


Definizione di funzione di interpretazione:
funzione che prende in input un linguaggio e lo
produce in un dominio.

\[ I_a: L_a \to D \]

Diciamo che l'asserzione $a$ denota il fatto $f$.


Non ambiguità: non è possibile che la funzione applicata
ad una asserzione ritorni fatti diversi. Se $f_1 \not = f_2$
sono diversi, non può essere che $I_a(a) = f_1$ e $I_a(a) = f_2$.

Polisemia (lingua italiana 2.3 in media ogni parola in italiano
ha più di 2 o poco più significati). Nel linguaggio naturale
tutto è informale, ambigua. Motivo per cui facciamo logica.

Sinonimia. No problem in logica $I_a(a_1) = I_a(a_2)$. Ma in
alcuni contesti può non essere desiderabile. In database, per esempio
questo non è ammesso per question di efficienza nella ricerca
di elementi.

Totalità. Funzioni di interpretazione sono totali, ogni elemento
della lingua ha un significato. wawawa non ha significato, perché
non appartiene al linguaggio. Elementi di linguaggio che non
denotano fatti -> brutti noooo.

Non Surgettività. Ma esiste un fatto per cui non esistono elementi del linguaggi
che lo denotano? Sì, è ammesso. Es persone in aula: creiamo
un linguaggio che descrive parte delle persone e degli oggetti
nell'aula. Perdiamo in completezza, ma funziona lo stesso.

Teoria massimale. è solo una, contiene una o più asserzioni per ogni fatto
nel modello inteso. Teoria in cui sai tutto.


\section{Modello del mondo}
tripla dove devo dare il linguaggio di asserzione il dominio di interpretazione
e una funzione di asserzione.

\[ W = \left\langle L_a, D, I_a \right\rangle  \]

Fatto ciò, notiamo che il mondo è una infrastruttura concettuale. La nostra
immagine che conosciamo in casi particolare è $M = \{f\} \subset D$.
Es database: con ER model ci veniva insegnato $W$, ma poi creavamo $M, T_a$.

\section{vero e falso formalmente}
Prendere la frase, interpretarla vedere cosa denota e vedere se è un fatto
ammissibile, che possiamo raggiungere, $f \in M$.



Esempio: parliamo del tempo di Trento, ma poi diciamo a new york piove. è vero
o falso. Nessuno dei due no, perché ogni asserzione è vera o falsa perché per
il nostro linguaggio è una parola ammissibile.

Disaccordo: funzione di interpretazione diversa, o modello inteso diverso
(eterogeneità semantica).


Gente parla di sintassi e semantica informalmente, intendendo sintassi
come sinonimo di grammatica.

\section{Corretto e completo}
Dato vero e falso, possiamo formalizzare correttezza e completezza.
Diciamo che dato un modello del mondo, un linguaggio asserzionale (che asserisce fatti sul mondo)

Correttezza linguaggio = tutte le asserzioni sono mappate in fatti del dominio.


Correttezza-completezza della teoria.


\section{Entailment}
Funzione fondamentale per ragionamento. Funz di interpretaz è $M \vDash T_a$

$I_a:$ dati gli elementi del linguaggio dice cosa vogliono dire


Abbiamo una foto con una montagna, una casetta, il sole (loda il sole) -> modello.
Prendiamo un sistema di computer vision che alleniamo a riconoscere
vari oggetti nelle immagini.
Teoria: "c'è una bicicletta" -> no non c'è, perché non sta nel modello.
Inoltre, per riconoscere quello che può esserci nel modello deve stare
nel linguaggio. Dico che $T$ è teoria se quello che tu mi hai detto c'è dentro il
modello.

Entailment è irrilevante sulla sintassi del linguaggio in uso, ciò che è
importante è la funzione di interpretazione, perché ci permette di validare
affermazioni etichettando con i concetti di vero e falso.


Applicazione di mondo aperto: (open world, in the wild) è tale per cui
a design time (quando scrivi software) non hai una specifica completa
del tuo input.














































RAgionamento> partire da premesse e arrivare a conclusioni.
Le premessse sono cose linguistiche che dici (quindi teoria),
conclusioni sono fatti. Quindi alla teoria aggiungiamo cose
nuove (unione di insiemi). Questo ovviamente non ha senso se
le premesse sono false. Abbiamo già formalizzato veirtà con
entailment $M \vDash T$.

Rappresentazione del mondo R coppia teoria-modello.

TellT restringe il numero di modelli compatibili. Più asserzioni
metto più aumentano i vincoli che impongo.

Indecidibile = tempo infinito per decidere









\section{Logica delle entità}
Facciamo una logica che ci faccia fare askc tellcassimassi sul modello più semplice (quale sarebbe???)

Data type è essenzialmente un etype, cioè un'entità che però non ``vediamo per strada'', 

Dominio è spazio di quello che possiamo vedere, quello che vediamo è il modello.


%%%%

Questa logica ha obiettivo formalizzare e automatizzare quel ragionamento che si fa nei world models.
Assumiamo che le entita abbiano un nome.

Slide which percepts: ha alfabeto che include ognuno dei sei percetti fondamentali. 

Fausto ha mangiato una mela, con soddisfazione, alle 5 pm, col coltello. Con loe rinunciamo
a relazioni narie, vogliamo solo quelle binarie (in realtà delle triple)


Alfabeto = insieme di parole (in realtà concetti, non parole nel senso del liguaggio naturale)
è na tripla

2 tipi di relazioni: proprietà-ruoli = relazioni tra etypes (padre-figlio) e relazioni

BNF = bachus naur form. $\not\mid$


$M \vDash T_a$ if and only if $I_(a) \in M$ for all $a \in T_a$

complex etypes